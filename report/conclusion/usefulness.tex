\section{Usefulness}

When hardware developers have access to tools like PET, the cost of doing energy
efficiency estimations will fall to a much lower level. It is not unusual to
have to go all the way to the RTL level and use some sort of low-level power
modeling such as SPICE \cite{ponomarev2002accupower} or PowerTheater
\cite{bruno2005rtl} to get values of decent accuracy. Wattch and McPAT has
provided some sort of easier access to architectural level power estimation, but
is still requires too much effort to be really easy in use. PET simply needs a
trace log from gem5, or with minor modifications, any application able to output
some sort of trace log. It seems that PET allows evaluation of the big picture much
earlier in the design stage than other existing options, simply because it is much
more agnostic regarding hardware details, thus we hope that PET will provide usefull
when developing both tiles for SHMAC and processors in general.

All in all, PET or other tools built from the same concept of weighting architectural
events are indeed possible for a set of scenarios, but it has not yet been validated for
any other architectures. The process of settings weights for PET seems cumbersome, but
for most practical settings the most important thing is to have the weights reasonably
propotioned amongst themselves.
