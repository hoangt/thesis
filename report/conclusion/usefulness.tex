\section{Usefulness}
When hardware developers have access to tools like PET, the cost of doing energy efficiency estimations will fall
to a much lower level. It is not unusual to have to go all the way to the RTL level and use some sort of
low-level power modeling such as SPICE \cite{ponomarev2002accupower} or PowerTheater \cite{bruno2005rtl} to
get values of decent accuracy. Wattch and McPAT has provided some sort of easier access to architectural level
power estimation, but is still requires too much effort to be really easy in use. PET simply needs a trace log
from gem5, or with minor modifications any application able to output some sort of trace log.

