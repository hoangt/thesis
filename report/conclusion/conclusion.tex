\section{Conclusion}

PET, a power estimation tool, is a software tool for estimating power
consumption on existing as well as non-existing computer architectures. It uses
output from gem5 together with a set of weighted parameters to estimate energy
consumption of a program running on a given hardware model. The weighted
parameters are selected by investigating the pipeline of an ARM~Cortex-A9
processor. We have run a set of workloads on the hardware platform and logged
their current drain over time. Further, the results were used as input for a
genetic algorithm that mapped the correct energy usage to each architectural
event in the simulator.

PET is not designed to be as accurate as possible, but to assist hardware
developers as early in the design stage as possible. As opposed to classical
methods, PET can be applied to a design already when only a simulation model
exists. Well known tools such as Wattch \cite{brooks2000wattch} or McPAT
\cite{li2009mcpat,li2013mcpat} also utilizes a simulator, but requires more knowledge about
the final hardware, e.g. RTL and process technology. Regardless of such
information, PET is able to estimate current drain with a margin of error within
5~\% when testing against the ARM~Cortex-A9 processor.

Because PET is a tool meant to be used early and rapidly in the design phase, it
has to be fast and easy to use. PET will predict power usage from log files and
is tested to evaluate about 133~MB of log files per second on a commodity
computer. Even with log files expanding tens of gigabytes, running PET takes
less time than running a low-level power estimator.

An excessive amount of time has been used for tweaking PET, gem5 and the genetic
algorithm to match our reference hardware platform as precisely as possible. Still,
we believe that the effort needed to port our methods to a new hardware platform
or architecture is much less. The genetic algorithm was in our case able to find
good weights within few hours, and with carefully selected power consuming
events it is likely that this is the same for other architectures. Unrealized
hardware still needs to derive weights from similar hardware.

Our observation is that PET allows evaluation of the big picture easier and
earlier in the design stage than existing solutions, simply because it
estimates power with less hardware details. We hope that PET will be useful
when developing both tiles for SHMAC architecture and processors in general.

All in all, using PET or other tools built from the same concept of weighting
architectural events is possible for a set of scenarios. How exact the model is
will depend on the simulator tuning and the genetic algorithm used to match
ammeter measurements. The process of settings weights for PET seems cumbersome,
but for most practical settings the most important thing is to have the weights
reasonably proportioned among themselves.
