\section{Problems of the New World}

Today's hardware designers are facing really hard problems. They must continue
to meet the market demand with respect to performance, while keeping the energy
consumption down. Energy consumption directly translates into heat generated
on-chip and is currently the limiting factor of processor design
\cite{patterson}. Dealing with this requires hardware designers to emphasize
energy efficiency over pure performance during the design phase; we are entering
an era where a processor's performance per Watt is more important than
performance itself.

Heat is not the only motivation factor to keep energy consumption down.
Processors targeting laptops, cellphones and other mobile devices has always
been energy-constrained due to their use of batteries. Lower energy consumption
would allow for longer battery life and/or heavier applications. Processors in
this segment trades off performance in return of increased energy efficiency.
More recently, however, mobile processors has become increasingly popular in
alternative domains too, such as supercomputing. Their low cost and good
performance per Watt ratio makes them attractive for massively parallel
problems, which is currently done on large and expensive supercomputers. These
machines have huge energy budgets and are taken out of service after
just a couple of years, being replaced by new machines that offer better
performance for less power. Building datacenters from low-cost
embedded processors is believed to have a huge potential and change the
landscape of supercomputing the years to come.
%TODO: Add cite

Not only data centers benefit from the use of mobile processors. The SHMAC
research project at NTNU aims to build a single-ISA heterogeneous computing
platform with processing cores tailored to the application. Using the most
effective processor or hardware accelerator -- in terms of both energy
and performance -- is the key to success for such platforms.

There are great reasons to minimize a processor's energy consumption. Batteries
would last longer, applications become richer and it will enable processor
performance increase to continue. Thus, performance alone is no longer the most
important attribute of processors, energy efficiency is even more important.


