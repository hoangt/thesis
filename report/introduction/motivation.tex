\section{Motivation for energy efficiency}
% Moore, blabla Amdahl, blabla Pollac, blabla other people.

During the 90's and early 2000's, single-threaded performance doubled every 18th
month. As more performance was crammed into larger and larger cores, heat and
power consumption became overwhelming. To remedy this, recent designs has put a
lot of effort into making the processor energy efficient.

Due to its battery-powered nature, mobile processors has always been
energy-constrained. Processors in this segment trades off performance in return
of increased energy efficiency. More recently, however, mobile processors has
become increasingly popular in alternative domains too, such as supercomputing.
Their low cost and good performance-per-watt ratio makes them attractive for
massively parallel problems: Building datacenters from low-cost embedded
processors is believed to have a huge potential and change the landscape of
supercomputing the years to come.

Not only datacenters benefit from the use of mobile processors. The SHMAC
research project at NTNU aims to build a single-ISA heterogeneous computing
platform with processing cores tailored to the application. Using the most
effective processor or hardware accelerator, in terms of both energy efficiency
and performance is crucial.

% TODO: Introduce the notion of 'energy profiles', i.e. energy/time graphs

