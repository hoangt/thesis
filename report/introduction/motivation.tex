\section{Motivation for Energy Efficiency}
% Moore, blabla Amdahl, blabla Pollac, blabla other people.

Ever since the birth of computers, the world has seen an ever-increasing demand
for computing power. Computers have become a vital part of the society,
supporting operations in government, health care and economics. Personal
computers have become ubiquitous and have found uses ranging from home
entertainment systems to portable tablet computers. As applications in all these
segments becomes richer, they demand better performance. Home users require
better performance on their multimedia applications, researchers can perform
more accurate simulations with more powerful computer systems and wireless
networks can increase connectivity with better encoding schemes.

To meet the increasing demand for performance throughout the 80's and 90's,
hardware designers were forced to optimize the processors designs. They were
able increase single-threaded performance tremendously by exploiting instruction
level paralellism, doubling the performance every 18th month. The tradeoff,
however, was added complexity to the core.


% 1. stoerre behov for performance
% 2. ikke lenger mulig å få masse performance
% 3. energi like viktig som performance (PPW)
% 4. energi er superviktig
% 5. Dennard scaling gjorde det lett å minimere energiforbruk
For a long time, new process technologies allowed for smaller and less energy
consuming transistors.


% alle datagrener har behov for stroemsparing, fordi: batteri, varme,
% driftkostnad, kombinasjon


As more performance was crammed into larger and larger cores, heat and
power consumption became overwhelming. To remedy this, hardware designers now
emphasize energy efficiency over pure performance; a processor's performance per
Watt is currently the measure designers seeks to maximize. Indeed, the overall
performance must also improve, forcing designers to employ novel techniques.

% vanskelig å oeke PPW -- mye utproevd allerede. PET gjoer forståelsen enklere

Mobile processors has always been energy-constrained due to their
battery-powered nature. Processors in this segment trades off performance in
return of increased energy efficiency. More recently, however, mobile processors
has become increasingly popular in alternative domains too, such as
supercomputing. Their low cost and good performance per Watt ratio makes them
attractive for massively parallel problems: Building datacenters from low-cost
embedded processors is believed to have a huge potential and change the
landscape of supercomputing the years to come.

% cite Tanenbaum -- tradisjonelt har mobil-PCer vært slappe og server high-end
% Top500: oeke cores for å minske power

Not only datacenters benefit from the use of mobile processors. The SHMAC
research project at NTNU aims to build a single-ISA heterogeneous computing
platform with processing cores tailored to the application. Using the most
effective processor or hardware accelerator -- in terms of both energy efficiency
and performance -- is the key to success for such platforms.

% SHMAC, power profiles. Velg beste kjerne for gitt workload/application

% TODO: Introduce the notion of 'energy profiles', i.e. energy/time graphs

