\section{Better Tools for Optimizing Energy Efficiency}
% Moore, blabla Amdahl, blabla Pollac, blabla other people.

Ever since the birth of computers, the world has seen an ever-increasing demand
for computing power. In present time, computers have become a vital part of the
society, supporting operations in government, health care and economics.
Personal computers have become ubiquitous and have found uses ranging from home
entertainment systems to portable tablet computers. As applications in all these
segments becomes richer, they demand even better performance. Home users require
higher frame rates on their multimedia applications, researchers can perform
more accurate simulations with more powerful computer systems and wireless
networks can increase connectivity with better encoding schemes.

To meet the increasing demand for performance throughout the 80's and 90's,
hardware designers were forced to optimize the processors designs. They were
able increase single-threaded performance tremendously by exploiting instruction
level parallelism and increasing frequency, doubling the performance every 18th
month. The trade-off, however, was added complexity to the processor core.
Greater complexity required a greater amount of transistors, so they had to be
crammed tightly together, which was made possible by the reduction in transistor
size. For a long time, new process technologies allowed for smaller and less
energy consuming transistors, but as we approached the end of Dennard scaling
\cite{dennard}, there was suddenly no free lunch. Heat generation on-chip
became overwhelming and one could not simply increase the frequency to gain
additional performance.

Today's hardware designers are faced against very hard problems. They must
continue to deliver performance boosts year after year, while maintaining
or even lowering the energy consumption. Performance alone is no longer the most
important attribute of processors, energy efficiency is even more important.

% alle datagrener har behov for stroemsparing, fordi: batteri, varme,
% driftkostnad, kombinasjon

To remedy this, hardware designers now emphasize energy efficiency over pure
performance; a processor's performance per Watt is currently the measure
designers seeks to maximize. Indeed, the overall performance must also improve,
forcing designers to employ novel techniques.

% vanskelig å oeke PPW -- mye utproevd allerede. PET gjoer forståelsen enklere

Mobile processors has always been energy-constrained due to their
battery-powered nature. Processors in this segment trades off performance in
return of increased energy efficiency. More recently, however, mobile processors
has become increasingly popular in alternative domains too, such as
supercomputing. Their low cost and good performance per Watt ratio makes them
attractive for massively parallel problems: Building datacenters from low-cost
embedded processors is believed to have a huge potential and change the
landscape of supercomputing the years to come.

% cite Tanenbaum -- tradisjonelt har mobil-PCer vært slappe og server high-end
% Top500: oeke cores for å minske power

Not only data centers benefit from the use of mobile processors. The SHMAC
research project at NTNU aims to build a single-ISA heterogeneous computing
platform with processing cores tailored to the application. Using the most
effective processor or hardware accelerator -- in terms of both energy
and performance -- is the key to success for such platforms.

% SHMAC, power profiles. Velg beste kjerne for gitt workload/application

% TODO: Introduce the notion of 'energy profiles', i.e. energy/time graphs

