% Moore, blabla Amdahl, blabla Pollac, blabla other people.

\section{Historical Perspective}

Ever since the birth of computers, the world has seen an ever-increasing demand
for computing power. In present time, computers have become a vital part of the
society, supporting operations in government, health care and economics.
Personal computers have become ubiquitous and have found uses ranging from home
entertainment systems to portable tablet computers. As applications in all these
segments becomes richer, they demand even better performance. Home users require
higher frame rates on their multimedia applications, researchers can perform
more accurate simulations with more powerful computer systems and wireless
networks can increase connectivity with better encoding schemes.

To meet the increasing demand for performance throughout the 80's and 90's,
hardware designers were forced to optimize the processors designs. They were
able increase single-threaded performance tremendously as displayed in \autoref{fig:moore} by exploiting instruction
level parallelism and increasing frequency, doubling the performance every 18th
month. The trade-off, however, was added complexity to the processor core.
Greater complexity required a greater amount of transistors, so they had to be
crammed tightly together, which was made possible by the reduction in transistor
size. For a long time, new process technologies allowed for smaller and less
energy consuming transistors, but as we approached the end of Dennard scaling
\cite{dennard}, there was suddenly no free lunch. Heat generation on-chip
became overwhelming and one could not simply increase the frequency to gain
additional performance.

\begin{figure}
\includegraphics[width=\textwidth]{analyze-spec-benchmarks/int_graph.png}
\caption{Single-threaded performance relative to the SUN Ultra Enterprise 2, recreated from \cite{preshing}}
\label{fig:moore}
\end{figure}

\section{Problems of the New World}

Today's hardware designers are facing really hard problems. They must
continue to deliver performance boosts year after year, while maintaining
or even lowering the energy consumption. Performance alone is no longer the most
important attribute of processors, energy efficiency is even more important.

% alle datagrener har behov for stroemsparing, fordi: batteri, varme,
% driftkostnad, kombinasjon

To remedy this, hardware designers now emphasize energy efficiency over pure
performance; a processor's performance per Watt is currently the measure
designers seeks to maximize. Indeed, the overall performance must also improve,
forcing designers to employ novel techniques.

% vanskelig å oeke PPW -- mye utproevd allerede. PET gjoer forståelsen enklere

Mobile processors has always been energy-constrained due to their
battery-powered nature. Processors in this segment trades off performance in
return of increased energy efficiency. More recently, however, mobile processors
has become increasingly popular in alternative domains too, such as
supercomputing. Their low cost and good performance per Watt ratio makes them
attractive for massively parallel problems: Building datacenters from low-cost
embedded processors is believed to have a huge potential and change the
landscape of supercomputing the years to come.

% cite Tanenbaum -- tradisjonelt har mobil-PCer vært slappe og server high-end
% Top500: oeke cores for å minske power

Not only data centers benefit from the use of mobile processors. The SHMAC
research project at NTNU aims to build a single-ISA heterogeneous computing
platform with processing cores tailored to the application. Using the most
effective processor or hardware accelerator -- in terms of both energy
and performance -- is the key to success for such platforms.

% SHMAC, power profiles. Velg beste kjerne for gitt workload/application

% TODO: Introduce the notion of 'energy profiles', i.e. energy/time graphs

\section{Better Tools for Optimizing Energy Efficiency}

Given the advanced tools creating hardware present these days, it is more easy
than ever to change, model and simulate performance and behaviour. With these
possibilities, a tool that could estimate difference in energy efficiency given
a simulator model would be a great help in the design process. There already
exists some such solutions, the most known might be
Wattch\cite{brooks2000wattch}, McPAT\cite{hpmcpat,li2013mcpat} and
CACTI\cite{hpcacti}. The different existing solutions utilize different
parameters and specialize in slightly different tasks.

NTNU have a long tradition of using M5/gem5 for simulation purposes, and it
would be most beneficial for the SHMAC project if such solutions could be
applied to an unmodified version of gem5. Wattch seems to be a good fit, but it
works only with SimpleScalar\cite{wattchanalysis}. McPAT worked with the old
M5\cite{m5mcpatparser}, which joined with GEMS to become gem5\cite{gem5hipeac},
but seems to have problems with the latest versions of
gem5\cite{mcpatgem5problems}. CACTI is a system for understanding trade-offs
between power, area and performance within memory
systems\cite{hpcacti,muralimanohar2010memory}. Both McPAT and Wattch uses
models and methods found in CACTI\cite{li2009mcpat}.

The immediate lack of a system that is easy to use and easy to set up motivates
the creation of a new tool that works on a higher level. PET is such a tool,
which is able to estimate power usage per time for a given workload on a given
architecture.  It will use an energy metric profile estimated from similar
processors together with a simulator trace log to calculate energy usage.
Using this approach, PET will be able to detect if hardware modifications done
to the simulation level model will be beneficial in the realized hardware. PET
will also tell if a processor-implementation is more energy efficient than an
other given a specific workloads, thus it can help building workload optimized tiles
for the SHMAC project\cite{shmacwebpage}. Using the features of PET, one can also
adjust the energy metric profile and simulate power usage as if one component was cheaper
or more expensive in terms of energy. This will enable the hardware designers to understand
where optimizations are most beneficial.



% motivation: early design stage, check architectural differences using same characteristics
% or set different weights to spesific events in the hunt for which components that needs optimization
