\section{Motivation for energy efficiency}
% Moore, blabla Amdahl, blabla Pollac, blabla other people.

% alle datagrener har behov for stroemsparing, fordi: batteri, varme,
% driftkostnad, kombinasjon

During the 90's and early 2000's, single-threaded performance doubled every 18th
month. As more performance was crammed into larger and larger cores, heat and
power consumption became overwhelming. To remedy this, hardware designers now
emphasize energy efficiency over pure performance; a processor's performance per
Watt is currently the measure designers seeks to maximize. Indeed, the overall
performance must also improve, forcing designers to employ novel techniques.

% vanskelig å oeke PPW -- mye utproevd allerede. PET gjoer forståelsen enklere

Mobile processors has always been energy-constrained due to their
battery-powered nature. Processors in this segment trades off performance in
return of increased energy efficiency. More recently, however, mobile processors
has become increasingly popular in alternative domains too, such as
supercomputing.  Their low cost and good performance per Watt ratio makes them
attractive for massively parallel problems: Building datacenters from low-cost
embedded processors is believed to have a huge potential and change the
landscape of supercomputing the years to come.

% cite Tanenbaum -- tradisjonelt har mobil-PCer vært slappe og server high-end
% Top500: oeke cores for å minske power

Not only datacenters benefit from the use of mobile processors. The SHMAC
research project at NTNU aims to build a single-ISA heterogeneous computing
platform with processing cores tailored to the application. Using the most
effective processor or hardware accelerator -- in terms of both energy efficiency
and performance -- is the key to success for such platforms.

% SHMAC, power profiles. Velg beste kjerne for gitt workload/application

% TODO: Introduce the notion of 'energy profiles', i.e. energy/time graphs

