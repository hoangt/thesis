\section{Optimizing for Energy Efficiency}

Given the advanced tools used to support hardware design these days, it is more
easy than ever to model and simulate performance of an unimplemented
architecture. However, modeling power consumption is a much more elaborate
process: Current techniques works on a very low level and uses circuit-level
models to obtain energy metrics. This makes them accurate, but also heavy and
time consuming to simulate. Being able to rapidly prototype and visualize how
changes in the microarchitecture affects energy performance is a key to build
energy efficient hardware as well as an important metric at the design stage. Some
solutions already exists, but most of them inspect energy consumption at very
fine granularity and requires ASIC synthesis of HDL code to work. During the
design process, there is a great need for tools that help developers predict
the changes in power consumption when new features are implemented.

The immediate lack of a system that is easy to use and easy to set up motivates
the creation of a new high-level tool. We introduce PET; a tool that is able to
estimate power usage per time for a given workload on a given architecture. It
will use an energy metric profile estimated from similar processors together
with a simulator trace log to calculate energy usage. Using this approach, PET
will be able to detect if hardware modifications done to the simulation level
model will be beneficial in the realized hardware. PET will also tell if a
processor-implementation is more energy efficient than an other given a specific
workload, thus it can help building workload optimized tiles for the SHMAC
project \cite{shmacwebpage}. Using the features of PET, one can also adjust the
energy metric profile and simulate power usage as if one component was cheaper
or more expensive to use (in terms of energy). This will enable hardware designers to
understand which optimizations are most beneficial and identify possible routes
of exploration in their journey of processor energy optimization.

% SHMAC, power profiles. Velg beste kjerne for gitt workload/application

% TODO: Introduce the notion of 'energy profiles', i.e. energy/time graphs

% motivation: early design stage, check architectural differences using same
% characteristics or set different weights to spesific events in the hunt for
% which components that needs optimization
