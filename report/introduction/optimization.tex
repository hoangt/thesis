\section{Optimizing for Energy Efficiency}

Given the availability of advanced hardware design tools, it is possible to
model and simulate performance of an unimplemented architecture with decent
accuracy. However, modeling power consumption is a more elaborate process:
current techniques works on a low level and uses circuit-level models to obtain
energy metrics. This method makes them accurate, but also heavy and time
consuming. Being able to rapidly prototype and visualize how changes in the
microarchitecture affects energy performance is an advantage when designing
energy efficient hardware. Some solutions already exists
\cite{bruno2005rtl,ponomarev2002accupower}, but most of them inspect energy
consumption at a fine granularity and requires ASIC synthesis of HDL code.
During the design process, there is a great need for tools that help developers
predict the changes in power consumption when new features are implemented.

The immediate lack of a system that is easy to use and set up motivates the
creation of a new high-level tool. We introduce PET; a tool that is able to
estimate power usage over time for a given workload on a given architecture. It
will use an energy metric profile together with a simulator trace log to
calculate energy usage. Using this approach, PET will be able to detect if
hardware modifications done to the simulation level model will be beneficial in
the realized hardware. PET will also tell if a processor implementation is more
energy efficient than another given a specific workload. Thus, it can help
building workload optimized tiles for the SHMAC project \cite{shmacwebpage}.
Using PET, one can also adjust the energy metric profile and simulate power
usage with one component cheaper or more expensive to use in terms of energy.
This will enable hardware designers to understand which optimizations are most
beneficial and identify possible routes of exploration in their journey of
processor energy optimization.

