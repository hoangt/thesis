\section{Historical Perspective}

Computers have emerged in many roles in our society, and the demand for greater
computer resources is ever increasing. Throughout the '80s and '90s, the
increasing demand for performance was met by increasing the clock frequency.
Shortening the critical path and exploiting instruction level parallelism
allowed the CPU to run at higher clock speeds to improve throughput
\cite{tanenbaum1984structured}. Consequently, processor manufacturers were able
to double single-threaded performance approximately every 18th month
\cite{moore1965cramming}. The tradeoff, however, was an increased amount of
complex logic added to the processor core. Techniques such as pipelining,
superscalarity and out-of-order execution all improved performance by leveraging
the increased number of transistors \cite{patterson}. For a long time, new
process technologies allowed for smaller and less energy consuming transistors,
but as we approached the end of Dennard scaling
\cite{dennard1974design,esmaeilzadeh2011dark}, the amount of logic required to
accommodate speedups could not fit on the die due to thermal constraints. Heat
generation on-chip became overwhelming; one could no longer add extra logic and
increase the frequency to gain additional performance.

