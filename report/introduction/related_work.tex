
\subsection{Related Work}

Wattch \cite{brooks2000wattch} and McPAT \cite{hpmcpat,li2013mcpat} are two
approaches on energy modeling, but both work on a relatively low level.

NTNU have a long tradition of using M5/gem5 for simulation purposes, and it
would be most beneficial for the SHMAC project if such solutions could be
applied to an unmodified version of gem5. Wattch seems to be a good fit, but it
only works with SimpleScalar \cite{wattchanalysis}. McPAT worked with the old M5
\cite{m5mcpatparser}, which joined with GEMS to become gem5 \cite{gem5hipeac},
but seems to have problems with the latest versions of gem5
\cite{mcpatgem5problems}. CACTI is a system for understanding trade-offs between
power, area and performance within memory systems
\cite{hpcacti,muralimanohar2010memory}. Both McPAT and Wattch uses models and
methods found in CACTI \cite{li2009mcpat}.




\section{Related Work}

Butko et. al. \cite{butko2012accuracy} evaluated the accuracy of the gem5
simulator against a Snowball SKY-S9500-ULP-C01 containing a dual core 1GHz ARM
Cortex-A9.  Their work reports very good accuracy for benchmarking workloads,
with runtime accuracy reported between 1.23\% and 35.31\%.  Most of the tests
were well within 6\% mismatch, but the tests containing a high number of
L2-misses had a high runtime discrepancy between hardware and simulation. Butko et. al.
utilized the SimpleTiming CPU model and the basic memory system of gem5.

Pusdesris et. al. \cite{pusdesrissources} evaluated errors in full system
simulation using gem5, and found that with proper modeling of the CPU model
and memory system, they got a mean runtime error of 17\% when running the PARSEC
test suite and 13\% running SPEC CPU2006. Their research is based on a ARM
Versatile Express TC2 development board containing an ARM Cortex-A15, a modified
ARM Out-of-order CPU model for gem5 and some adjustments to the memory model.

This work is based on Runde et. al. \cite{rundehvatum2013exploring} where isolated
power drain of different instructions where measured.

\begin{enumerate}
    \item SimplePower (Penn State)
    \item McPAT (HP-Labs)
    \item PowerAnalyzer (University of Michigan)
    \item Wattch (Princeton)
    \item LLVM Energy Models
\end{enumerate}
