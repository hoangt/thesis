\section{Hardware Simulators}

As computer architecture development meets more challenging demands, a versatile
set of software tools have been developed to help the designers. In this set of
tools lies a set of computer architecture simulators meant to evaluate
processors at the architectural level.  They provide the ability to model
hardware at a higher abstraction layer than what's expressed by the underlying
circuit. When executing a binary program, they can output detailed trace logs of
the processor activity, together with performance data.  The benefits of
architectural simulators are many, but it is important to find one that fits the
demands given by the problem.

\subsection{A Brief Comparison}
\label{subsec:simulators}
There are numerous computer architecture simulators available. To support our
power estimation tool, a simulator frontend must provide a good picture of
events that occur in a hardware implementation of the architecture. The
out-of-order property significantly increases the level of complexity which the
simulator must handle. The following simulators include properties that can
support our approach:

\begin{description}
\item[Sniper] \hfill\\
    Sniper is a high-speed, multicore, multi-threaded and cycle-accurate
    computer architecture simulator \cite{sniperwebpage,carlson2013ssomta}. It
    already integrates with McPAT and it is open source. Sniper does only work
    x86, and is thus not applicable for simulation of ARM-based architectures.

\item[SimpleScalar] \hfill\\
    SimpleScalar is a popular commercial architectural simulator that comes with
    a free academic license providing full source code. SimpleScalar supports
    the ARM instruction set amongst many others, and looks like a decent
    simulator for advanced out-of-order core simulation. SimpleScalar is also
    the simulator used by the Wattch-project \cite{brooks2000wattch}. However,
    the SimpleScalar web page seems to be last updated in 2004 and all mailing
    list archives are gone at the time of writing. The source code for
    SimpleScalar v3 is still available, but has only been patched once since
    2003.

\item[QEMU]\hfill\\
    QEMU is a generic and open machine emulator which enables near real-time
    performance on architectures like ARM, even on x86 host machines. However,
    QEMU is a machine emulator rather than an architectural simulator, thus
    despite its great performance of running ARM-binaries, it will not produce
    CPU and memory event trace logs, and is not suitable for this project.

\item[gem5]\hfill\\
    gem5 is a merger between the M5 simulator \cite{binkert2006m5} and the GEMS
    simulator project \cite{GEMS}. gem5 includes ARM-support with out-of-order
    execution out of the box, and provides great cycle-accurate trace logs which
    are appropriate for this project \cite{gem5simulator}. Its core is written
    in C++ and has a highly modular interface that allows users to specify
    simulator target through simple Python scripts. Many of the maintainers are
    employees of ARM Corp., and the activity on the mailing lists suggests high
    project activity \cite{gem5dev}.
\end{description}

Provided this comparison of simulators, and given the fact that NTNU have
experience with gem5, gem5 is the natural winner and our choice of an
architectural simulator.


\subsection{The gem5 Simulator}

The gem5 project \cite{gem5} merges the best features of M5 \cite{binkert2006m5}
and GEMS \cite{GEMS} and includes a wide range of CPU and memory models
\cite{gem5hipeac}. The simulator has two main execution modes; \textit{Syscall
Emulation} (SE) or \textit{Full System} (FS). In SE mode, the simulator traps
system calls in the binary and emulates them, often by passing them to the host
operating system. In FS mode, the simulator can load an entiry operating system
binary (e.g. a GNU/Linux distribution) and run applications within the OS. gem5
supports many architectures; it can run binaries compiled for ALPHA, SPARC,
MIPS, ARM, x86 and POWER architectures.

During simulation, gem5 keeps track of hundreds of events related to the CPU
core and memory system. In-detail statistics, similar to performance counters on
real hardware, are then dumped for subsequent inspection. gem5 is also able to
output a trace log while it runs, originally intended for debugging of gem5.A
trace log contains user-selected events that happens within the simulated
hardware. These trace logs grow quickly in size, easily 10s of gigabytes, but
provides useful insights of the simulated execution. In particular, they describes
CPU activity down to the microarchitecture level and outputs simulated processor
activity.

