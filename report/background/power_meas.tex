\section{CPU-level Energy Measurements}

High quality instruction level energy models can be derived for pipelined
processors by monitoring the instantaneous current drawn by the processor at
each clock cycle \cite{nikolaidis2005instruction}. As modern processors commonly
operate at a few GHz, expensive measurement devices are required to sample
at sufficient frequency.

In \cite{rundehvatum2013exploring}, we conducted experiments to quantify the
energy cost of an instruction executing on a modern out-of-order mobile
processor while sampling at significantly lower rate. We were able to do this by
completely bypassing the memory hierarchy utilizing special hardware
(\emph{fast-loop mode}) and sampling a running average. Voltage drop over a
shunt resistor in series with the ODROID-X2 development board is measured, and
we calculate energy used in the processor core. We isolate the energy used in
the CPU (in which we are interested in) with energy used for on-board
peripherals by modifying the development board and driving the CPU core with a
separate power supply.

To derive the energy used in the processor, we first calculate the amount of
current flowing through the circuit. We assume that the shunt resistor voltage
drop is negligible compared the voltage drop over the ODROID, so that we have
1.3V on both sides of the resistor. Ohms law expresses current as

\begin{equation}
    I=\frac{U}{R}
    \label{eq:ohm}
\end{equation}

The voltage $U$ is being measured, while the restistance $R$ is a known fixed
value (12 m$\Omega$ in our experiments). Relating current to energy consumption
is done by applying the definition of electric power.

\begin{equation}
    P=U \cdot I
    \label{eq:power}
\end{equation}

Now, the current through the shunt resistor is the same as through the CPU core,
and the voltage is assumed to a fixed 1.3V.

% TODO: Where does this fit?
% The ultimate goal in this context is to estimate power consumption and energy
% efficiency of new hardware, the first step is to measure different more or less
% power consuming events on real hardware that is similar to the one simulated.

