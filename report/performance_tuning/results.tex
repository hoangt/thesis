\section{Results}



There are several factors that together makes up the total performance of PET. First, the simulator must be
configured as close to the real hardware as time and knowledge permits. Next, one must find power weights that
seems close to those of the realized hardware. In this particular research, the processor is already existing,
and the simulator, PET and the strategy of the gentetic algorithm will determine how close power consumption
can be estimated. This section will take a closer look to each of these parameters.

\subsection{gem5 CPU model accuracy}

After modeling different configurations for gem5, accuracy as depicted in \autoref{tbl:gem5runtimeaccuracy}
was achieved. Each test was run with the command written out in \autoref{lst:gem5commandline}, with \texttt{CPU}
changed with  \texttt{exynos\_4412p}, \texttt{arm\_detailed} and \texttt{timing}.

% Move table to result chapter
\begin{table}
\centering
\begin{tabular}{|l|c|c|c|c|c|}
\hline
   & add-add & pi-pi & sha2-sha2 & trend-trend & trend-submul\\
\hline
Real hardware & 0.014900  & 0.013500 & 0.022600 & 0.014600 & 0.028200\\
gem5 modified O3    & 0.017541 & 0.013790 & 0.022819 & 0.011898 & 0.023978 \\
gem5 original O3    & 0.008777 & 0.006708 & 0.010368 & 0.004344 & \\
gem5 timing simple  & 0.035039 & 0.019564 & 0.040659 & 0.020503 & \\
\hline
\end{tabular}
\caption{gem5 runtime accuracy (O3 with classic memory system)}
\label{tbl:gem5runtimeaccuracy}
\end{table}

\begin{lstlisting}[float=htb,language=sh,numbers=none,label={lst:gem5commandline},caption={gem5 Command Line}]
$ build/ARM/gem5.opt --remote-gdb-port=0 -d m5out-time/sha2-sha2
    configs/example/se.py -c bin/sha2/sha2 --cpu-type=CPU
    --mem-type=LPDDR2_S4_800_x32 --sys-clock=440MHz
    --cpu-clock=1700MHz --num-l3caches=0 --caches --l2cache
    --l2_assoc=16 --l2_size=1MB --l1d_size=32kB
    --mem-size=2048MB --l1d_assoc=4 --l1i_assoc=4
\end{lstlisting}

\subsection{GA optimization and results}

As PET has been optimized by a genetic inspired algorithm, and it was expected
to perform very well on the data sets it has been optimized for. When such
training is done, a controll test is needed in order to verify that the result
is good for the general problem solved, not only the specific instances used for
training.


Results from the training session arer displayed in
\autoref{fig:trend-training}, \autoref{fig:pi-training},
\autoref{fig:sha2-training} and \autoref{fig:add-training}. Each figure
represents one of the workloads explained in \autoref{sec:workloads}.
The red line in each figure represents the prediction done by PET, while
the green line is a direct plot of readings from the bench setup (see \autoref{fig:setup} for a recap).

\begin{figure}[htb]
    \centering
    \includegraphics[width=\textwidth]{figs/training/trend-trend.pdf}
    \caption{Training results for the trend-test}
    \label{fig:trend-training}
\end{figure}

The training set for \emph{trend} as printed in \autoref{fig:trend-training}.
This test checks if PET can follow the flow from a ALU-intensive program to a
memory-intensive program. The graph shows that PET misses with a little more
than 10mA, but the most important factor is that the change after about 7ms is
resembled in the prediction. Given this change, we know that PET adjusts
correctly for the change in instruction flow. The fact that the sudden drop in
power consumption seems to come a bit too slow is most likely due to
discrepancys between the gem5-model and the real hardware. It is also
interesting that the begining and the end of the program is the most power
consuming pieces. This is probably due to the simplistic method idle-time is
calculated, and the out-of-order-nature of the processor makes the genetic
algorithm over-estimate the cost of idle.

\begin{figure}[htb]
    \centering
    \includegraphics[width=\textwidth]{figs/training/trend-submul.pdf}
    \caption{Overlay of PET training results (red) and training data (green)}
    \label{fig:submul-training}
\end{figure}

Further, the \emph{submul}-test shown in \autoref{fig:submul-training} is
created to follow the shift from simple integer operations and over to
a multiplication-dominated loop. PET again follows the trend, but is
overly hard on the cost of multiply. Again, the trend is the most important
factor. It should also be noted that even tough the loop contains a lot of
multiply, the assembled binary still uses a lot of "ordinary" integer
operations for program flow, storage, registers, adresses, and so on.

\begin{figure}[htb]
    \centering
    \includegraphics[width=\textwidth]{figs/training/pi-pi.pdf}
    \caption{Overlay of PET training results (red) and training data (green)}
    \label{fig:pi-training}
\end{figure}

The \emph{pi}-test displayed in \autoref{fig:pi-training} uses floating point
and random-values (with a static seed).  the program is compiled with
soft-float, thus it makes just another integer-test, but with a more intense use
of different registers. This test has rather good accuracy, and as it represents
a rather common workload, this is a very promissing result.

\begin{figure}[htb]
    \centering
    \includegraphics[width=\textwidth]{figs/training/sha2-sha2.pdf}
    \caption{Overlay of PET training results (red) and training data (green)}
    \label{fig:sha2-training}
\end{figure}

The last test in the performance tuning algorithm, \emph{sha2}, calculates a
sha512 hash sum.  The test results is figured in \autoref{fig:sha2-training} and
differes from the other tests by beeing more memory intensive at the same time
as it is an extensive user of both multiply and simple integer operations. This
test is also well within 10\% of measured and is yet another example of good
performance on general workloads.

