\section{Choosing Workloads}
\label{sec:workloads}

When running a genetic algorithm, it is critical to lead the evolution in the
correct direction. In our case, this is done by providing a reasonable set of
workloads (i.e. ARMv7 programs) that stresses distinct modules in the processor.
For instance, a memory intensive workload will have high density of
memory-related events from the simulator, and will support the genetic algorithm
in determining cost for memory accesses. It is important for the set of
workloads that are chosen to be diverse and stress many conditions the processor
can operate in, e.g. mixes of compute intensive and memory intensive programs. A
poorly chosen set of workloads will not give a fair judgement on which genomes
that fit well. A bad workload might be to biased towards a single or a subset of
parameters, neglecting the rest, or even mislead the GA into a local optimum
\cite{introtoga}. Another worry is that within the training set, there will most
likely exist multiple Pareto optimal solutions \cite{deb2014multi}, but only one
of these can truly match the real power consumption.

We came up with the following four workloads.

\begin{description}
    \item[Pi] \hfill \\
        This test calculates Pi using Monte Carlo simulation. It includes
        floating point and more advanced operations such as multiply. It runs
        for a fixed amount of iterations.
    \item[SHA-512] \hfill \\
        The SHA-512 algorithm is a hashing algorithm used in cryptography. It
        includes a mix of integer operations and memory usage. Source code
        fetched from sha2-1.0.tgz at \cite{sha2}.
    \item[Trend] \hfill \\
        This test has two parts. It starts with a tight add loop, and then
        continues with extensive memory allocation. Presumably, this will create
        a shift in energy consumption between the two stages.
    \item[SubMul] \hfill \\
        The SubMul test borrows ideas from the previous program, but instead of
        testing ALU and memory, this test compares subtract and multiply.
\end{description}

We claim that the workloads used in this experiment spans the most common
instruction types while being simple enough to be simulated in gem5 on
reasonable time.

