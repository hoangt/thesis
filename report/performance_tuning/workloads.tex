\section{Choosing Workloads}
\label{sec:workloads}

When running a genetic algorithm, it is critical to lead the evolution in the
correct direction. In our case, this is done by providing a reasonable set of
workloads (i.e. ARMv7 programs) that stresses distinct modules in the processor.
For instance, a memory intensive workload will have high density of
memory-related events from the simulator, and will support the genetic algorithm
in determining cost for memory accesses. It is very important for the set of
workloads that are chosen to be diverse and stress many conditions the processor
can operate in, e.g. mixes of compute intensive and memory intensive programs.
A poorly chosen set of workloads will not give a fair judgement on which genomes
that fit fit well. A bad workload might be to biased towards a single or a set
of parameters, neglecting the rest, or even mislead the genetic algorithm into a
local minimum \cite{introtoga}. Another worry is that within the training set,
there will most likely exist multiple Pareto-optimal solutions
\cite{deb2014multi}, but no more than one of these can truly match the real
power consumption.

We came up with the following four workloads. The binary name is is the
parenthesis.

\begin{description}
    \item[Pi (pi-pi)] \hfill \\
        This test calculates Pi using the Monte-Carlo method. It includes
        floating point and more advanced operations such as multiply.
    \item[Trend (trend-trend)] \hfill \\
        This test resembles a two-part program consisting of
        one part with ALU-operations and one part with intense memory usage.
    \item[Submul (trend-submul)] \hfill \\
        The submul test borrows ideas of a trend change in the middle of
        the program, but instead of testing memory and ALU, this test compares
        subtract and multiply.
    \item[SHA-512 (sha2-sha2)]
        \footnote{Despite the name, this implementation runs a SHA-512
        algorithm} \hfill \\
        The SHA-512 algorithm includes a mix of integer operations and
        memory usage. Source code fetched from sha2-1.0.tgz at \cite{sha2}.
\end{description}

We claim that the workloads used in this experiment spans the most common
instruction types while being simple enough to be simulated in reasonable time.

