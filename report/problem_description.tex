\begin{titlingpage}

\noindent
\begin{tabular}{@{}p{4cm}l}
    \textbf{Title:}     & \thetitle \\
    \textbf{Students:}  & Terje Runde \& Stian Hvatum \\
\end{tabular}

\vspace{4ex}
\noindent\textbf{Problem description:}
The SHMAC prototype is an ongoing research project within the Energy
 Efficient Computing Systems (EECS) strategic research area. SHMAC is planned
 to run in an FPGA and be an evaluation platform for research on
 heterogeneous multi-core systems. Due to the Dark silicon effect, future
 computing systems are expected to be power limited. The goal of the SHMAC
 project is to propose software and hardware solutions for future
 power-limited heterogeneous systems.

The micro architecture level is an implementation of the Instruction Set
 Architecture (ISA). Energy efficiency of an ISA is as such given by the
 chosen micro architecture. To be able to take the "right" design choice to
 optimize for energy efficiency, knowledge of energy and power for
 instruction types, e.g. instructions of type float, nop, copy, are needed.

The goal of this sub-project within the SHMAC platform is to gain knowledge
 of energy/power consumption of different instruction types to be able to
 extract information that can be used to improve the micro architecture
 design of SHMAC-cores. This project will take a twofold approach; 1)
 Investigate the power/energy consumption of simple benchmark programs on
 real hardware, i.e. create benchmark programs and evaluate performance by
 measurements. 2) Investigate the same benchmark programs in simulations as
 to ensure a good understanding of the relation between measurements and
 simulated results.

\noindent The project will include:
\setitemize[0]{topsep=0pt,partopsep=0pt,parsep=0pt,itemsep=0pt}
\begin{itemize}
    \item Devising small benchmark programs, e.g. C or assembly, that isolate specific functions at the micro architecture level.
    \item Run test on real hardware to collect data.
    \item Run tests in simulation to relate measurements to simulation results.
\end{itemize}

\noindent An ARM processor is going to be the target ISA for measurements and
simulations.
\vspace{6ex}

\noindent
\begin{tabular}{@{}p{4cm}l}
    \textbf{Responsible professor:} & \theprofessor \\
    \textbf{Supervisor:}            & \thesupervisor \\
    \textbf{Cosupervisor:}          & \thecosupervisor \\
\end{tabular}

\end{titlingpage}
