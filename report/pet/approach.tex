\section{Approach}

There are many considerations to take before creating a tool that should pretend
to understand the implementation of hardware and the implications of features
regarding energy efficiency. Care must be taken so that important information is
extracted from the hardware model while irrelevant noise is filtered out.


\subsection{Energy Modeling}

Song et. al \cite{song2012instruction} identifies three major approaches to
processor power modeling used in the past, and introduces an instruction-based
energy estimation model that can be used for energy simulation at high speed.
Their proposed method is expressed through the following equation and includes
the desired features of past energy models.

\begin{equation}
    P_{core}(t) = \frac{E_{unit} \cdot A_{datapath} \cdot w(t) +
    E_{static}}{T_{sampling}}
\end{equation}

This method depends on two things. First, one must know sufficient details of
the processor to identify datapath components in order to form the
$A_{datapath}$ matrix. The entries in $A_{datapath}$ are the invocation counts
of physical components in the datapath with respect to the workload metric
$w(t)$. $w(t)$ is typically comprised of instruction types or key operational
parameters such as cache miss, ratio, pipeline stall cycles and number of
executed instructions. Secondly, the energy unit vector $E_{unit}$, a vector
enumerating the per-access energy cost, requires circuit-level knowledge of the
target processor to calculate. $A_{datapath}$ can often be found by reverse
engineering and benchmarking, however, the $E_{unit}$ is rarely available for
commercial processors. When building the model for PET, we simplify the model
from \cite{song2012instruction} by combining $A_{datapath}$ and $E_{unit}$ to
form a vector of weights that directly corresponds to the cost of an event. We
can then model energy by the following formula.

\begin{equation}
    P_{core}(t) = \frac{C \cdot w(t) + E_{static}}{T_{sampling}}
\end{equation}

Here, C represents the global cost vector -- a matrix enumerating the cost
for all event types. Note that it is global and do not depend on time.

\subsection{Power Consuming Events}
\label{subsec:powerevents}

Choosing which events should be tracked and what workload that would give good
metrics is an important part of our method. We account for two main groups of
events; CPU instruction events and memory activity events. The contents of these
groups are listed in \autoref{tbl:events}. It is desirable to estimate energy
consumption on literally all types of computing systems, ranging from large-size
clusters to embedded systems. To provide this flexibility it was decided that
PET should parse log files from the simulators rather than being built-in on a
specific simulator. Most active and working architectural simulators supports this
sort of trace logs, and even if they are formatted different, the effort of
adjusting to a new format is a lot less than the effort of building this tool
within another simulator.

The trace logs contains information about everything that goes on within the
fictional computer, and thus PET can extract useful information from this log
file. A piece of such useful information is defined in PET as a \emph{simulator
event}. A simulator event can be thought of as a unit of work that uses a
specified amount of energy. When PET finds such an event, it increases the
modelled energy consumption along the timeline at the time the event took place.

\begin{table}[ht]
    \centering
    \begin{minipage}[b]{\linewidth}
        \centering
        \begin{tabular}{|l|l|}
            \hline
            IntAlu    & Integer basic ALU operation\\
            \hline
            IntMult    & Integer multiply ALU operation \\
            \hline
            MemRead    & Memory Read issued, triggers LS-unit \\
            \hline
            MemWrite    & Memory Write issued, triggeres LS-unit \\
            \hline
            SimdFloatMisc     & NEON-unit activated \\
            \hline
        \end{tabular}
        \subcaption{CPU Core Events.}
    \end{minipage}

    \begin{minipage}[b]{\linewidth}
        \centering
        \begin{tabular}{|l|l|}
            \hline
            L1IR    & L1 instruction cache, read \\
            \hline
            L1IW    & L1 instruction cache, write \\
            \hline
            L1DR    & L1 data cache, read \\
            \hline
            L1DW    & L1 data cache, write \\
            \hline
            L2R     & L2 cache, read \\
            \hline
            L2W     & L2 cache, write \\
            \hline
            PhysR   & Main memory, read \\
            \hline
            PhysW   & Main memory, write \\
            \hline
        \end{tabular}
        \subcaption{Memory Events.}
    \end{minipage}
    \caption{Power Consuming Events.}
    \label{tbl:events}
\end{table}

The events briefly described in \autoref{tbl:events} are the ones recognized by
PET. These events are selected based mainly on what information that is easily
extracted from a gem5-formatted trace log, but also adjusted by what information
we could check with performance counters without a very high amount of effort.
Most of this information is available from \cite{rundehvatum2013exploring},
where different instruction loops where measured with both ammeter and
performance counters. This is then correlated with the workings of
the pipeline (as seen in \autoref{fig:a9arch}).

