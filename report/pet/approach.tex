\section{Approach}

There are many considerations to take before creating a tool that should pretend
to understand the implementation of hardware and the implications of features
regarding energy efficiency. Through the next sections, we a model to be
used in PET is developed and its inputs are defined.


\subsection{Energy Modeling}

Song et. al \cite{song2012instruction} identifies three major approaches to
processor power modeling used in the past, and introduces an instruction-based
energy estimation model that can be used for energy simulation at high speed.
Their proposed method is expressed through the following equation, and includes
the desired features of previous energy models.

\begin{equation}
    P_{core}(t) = \frac{E_{unit} \cdot A_{datapath} \cdot w(t) +
    E_{static}}{T_{sampling}}
\end{equation}

This method has two dependencies. First, one must have sufficient details of the
processor in order to identify datapath components to form the $A_{datapath}$
matrix. The entries in $A_{datapath}$ are the invocation counts of physical
components in the datapath with respect to the workload metric $w(t)$. $w(t)$ is
typically comprised of instruction types or key operational parameters such as
cache miss, ratio, pipeline stall cycles and number of executed instructions.
Secondly, the energy unit vector $E_{unit}$, a vector enumerating the per-access
energy cost, requires circuit-level knowledge of the target processor to
calculate. $A_{datapath}$ can often be found by reverse engineering and
benchmarking. The $E_{unit}$, however, is rarely available for commercial
processors.

When building the model for PET, the model from \cite{song2012instruction} is
simplified by combining $A_{datapath}$ and $E_{unit}$ to form a vector of
weights that directly corresponds to the cost of an event. We call this vector
$C$. Power for each core over time, $P_{core}(t)$, is then modeled by the
following formula.

\begin{equation}
    P_{core}(t) = \frac{C \cdot w(t) + E_{static}}{T_{sampling}}
\end{equation}

Here, $C$ represents the global cost vector -- a matrix enumerating the cost
for all event types. Note that it is global and do not depend on time.
$T_{sampling}$ represents the sampling period and $E_{static}$ the static energy
consumption.

\subsection{Power Consuming Events}
\label{subsec:powerevents}

Choosing which events should be tracked and which workloads that would give good
metrics is an important part of our method. We account for two main groups of
events; CPU instruction events and memory activity events. The events in these
groups are listed in \autoref{tbl:events}. It is desirable to estimate energy
consumption on literally all types of computing systems, ranging from large-size
clusters to embedded systems. To provide this flexibility it was decided that
PET should parse log files from the simulators rather than being built-in on a
specific simulator. Most active and working architectural simulators supports this
sort of trace logs. Even if they are formatted differently, the effort of
adjusting to a new format is a lot less than the effort of building this tool
within a simulator.

\begin{table}[ht]
    \centering
    \begin{minipage}[b]{\linewidth}
        \centering
        \begin{tabular}{|l|l|}
            \hline
            IntAlu      & Basic integer ALU operation\\
            \hline
            IntMult     & Integer multiply ALU operation \\
            \hline
            MemRead     & Memory Read issued, triggers LS unit \\
            \hline
            MemWrite    & Memory Write issued, triggers LS unit \\
            \hline
            SimdFloatMisc     & NEON unit activated \\
            \hline
        \end{tabular}
        \subcaption{CPU core events.}
    \end{minipage}

    \begin{minipage}[b]{\linewidth}
        \centering
        \begin{tabular}{|l|l|}
            \hline
            L1IR    & L1 instruction cache, read \\
            \hline
            L1IW    & L1 instruction cache, write \\
            \hline
            L1DR    & L1 data cache, read \\
            \hline
            L1DW    & L1 data cache, write \\
            \hline
            L2R     & L2 cache, read \\
            \hline
            L2W     & L2 cache, write \\
            \hline
            PhysR   & Main memory, read \\
            \hline
            PhysW   & Main memory, write \\
            \hline
        \end{tabular}
        \subcaption{Memory events.}
    \end{minipage}
    \caption{Power consuming events.}
    \label{tbl:events}
\end{table}

The trace logs contains information about everything that goes on within the
fictional computer. Such a piece of information is defined in PET as a
\emph{simulator event}. A simulator event can be thought of as a unit of work
that uses a specified amount of energy. When PET finds such an event, it
increases the modeled energy consumption at the correct point in time where the
event took place.

The events described in \autoref{tbl:events} are the ones currently recognized
by PET, but adding more (or removing) events is trivial. These events are
selected mainly based on the information which is easily extracted from a
gem5-formatted trace log, but also adjusted according to what could be checked
with performance counters on the target hardware. Most of this information is
available from \cite{rundehvatum2013exploring}, where different instruction
loops were measured with both ammeter and performance counters. This is then
correlated with the properties of the pipeline (as seen in
\autoref{fig:a9arch}).

