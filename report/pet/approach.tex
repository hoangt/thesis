\section{Approach}

While computer architecture simulators have been around for quite some time,
energy estimation techniques is a more recent necessity. Song et. al
\cite{song2012instruction} identifies three major approaches to processor power
modelling used in the past, and introduces an instruction-based energy
estimation model that can be used for energy simulation at high speed. Their
proposed method is expressed through the following equation and includes the
desired features of past energy models.

\[
    P_{core}(t) = \frac{E_{unit} \cdot A_{datapath} \cdot w(t) +
    E_{static}}{T_{sampling}}
\]

This method depends on two things. First, one must know sufficient details of
the processor to identify datapath components in order to form the
$A_{datapath}$ matrix. Secondly, the energy unit vector $E_{unit}$ requires
circuit-level knowledge of the target processor. The former information can
often be found by reverse engineering and benchmarking, however, the latter is
rarely available for commercial processors. When building the model for PET, we
simplify the model from \cite{song2012instruction} by combining $A_{datapath}$
and $E_{unit}$ to form a vector of weights that directly corresponds to the cost
of an event. We can then model energy by the following formula.

\[
    P_{core}(t) = \frac{C \cdot w(t) + E_{static}}{T_{sampling}}
\]

Here, C represents the global cost vector -- a matrix enumerating the cost
for all event types. Please note that it is global and do not depend on time.

The choice of events/workload metrics is an important part of our method. We
account for two types of events; CPU instruction events and memory activity
events.

A major design choice made on how PET should be designed was to decide whether
it should be a standalone program or incorporated into an existing simulator.
There already exists a lot of computer architecture simulators, all with
strengths and weaknesses. Often, simulators only supports a small set of
architectures, memory systems and CPU models, and they only excel in simulating
one specific combination.


\subsection{Power Consuming Events}
\label{subsec:powerevents}
It is desirable to estimate energy consumption on literally all types of
compututing systems, ranging from large-size clusters to embedded systems. To
provide this flexibility it was decided to write the tool as simulator agnostic
as possible, but still track \emph{simulator events} rather than executed instructions.

A simulator event is defined as a unit of work that uses a specified amount of
energy, and increases modelled energy consumption to the simulated time. We
defined a set of simulator events that PET should consider. The selected events
are briefly described in \autoref{tbl:events}.

\begin{table}[ht]
    \centering
    \begin{minipage}[b]{\linewidth}
        \centering
        \begin{tabular}{|l|l|}
            \hline
            IntAlu    & Integer basic ALU operation\\
            \hline
            IntMult    & Integer multiply ALU operation \\
            \hline
            MemRead    & Memory Read issued, triggers LS-unit \\
            \hline
            MemWrite    & Memory Write issued, triggeres LS-unit \\
            \hline
            SimdFloatMisc     & NEON-unit activated \\
            \hline
        \end{tabular}
        \subcaption{CPU Core Events}
    \end{minipage}

    \begin{minipage}[b]{\linewidth}
        \centering
        \begin{tabular}{|l|l|}
            \hline
            L1IR    & L1 instruction cache, read \\
            \hline
            L1IW    & L1 instruction cache, write \\
            \hline
            L1DR    & L1 data cache, read \\
            \hline
            L1DW    & L1 data cache, write \\
            \hline
            L2R     & L2 cache, read \\
            \hline
            L2W     & L2 cache, write \\
            \hline
            PhysR   & Main memory, read \\
            \hline
            PhysW   & Main memory, write \\
            \hline
        \end{tabular}
        \subcaption{Memory Events}
    \end{minipage}
    \caption{Power Consuming Events}
    \label{tbl:events}
\end{table}

This 

gem5 was used as a basis for development and the only simulation front-end
implemented.
