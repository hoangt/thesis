\section{Methodology}

\subsection{Design Choices}
\label{subsec:design_choises}
The first choices made on how PET should be designed was to decide whether it
should be a standalone program or incorporated into an existing simulator.
Hundreds of computer architecture simulators exists, all with strengths and
weknesses. Often, simulators only supports a small set of architectures, memory
systems and CPU models, and they are only good at simulating one specific
combination.

It is desirable to estimate energy consumption on literally all types of
compututing systems, ranging from large-size clusters to embedded systems. To
provide this flexibility it was decided to write the tool as simulator agnostic
as possible, tracking \emph{simulator events} rather than executed instructions.
A simulator event is defined as a unit of work that uses a specified amount of
energy, and increases modelled energy consumption to the simulated time.

%\begin{center}
%    \begin{tabular}{|l|l|}
%        \hline
%        L1D & L1 data hit\\
%        \hlinke
%    \end{tabular}
%\end{center}

gem5 was used as a basis for development and the only simulation front-end
implemented.




- ARM support (Sniper has not)

QEMU -- feil abstraksjonsnivå/formål
McPat - multicore, power AND AREA???? Optimizer
gem5 is flexible because
    - it is easy to add new processor architectures
    - memory system
=> in-house experience

LLVM traces?

