\section{Concept}
The goal for PET and this project is to ultimatly estimate power usage and thus
also energy efficiency of still not yet implemented computer architectures. A
such approach will of cource reside well within the term "estimation" and will
doubtly reach the correct numbers. Never the less, PET is built by measuring
real hardware with great detail, capturing discrete events and assigning each
event a certain amount of energy consumption.

When selected events have been weighted, one can run a simulation of similar
hardware, retreive a tracelog containing these events, and then apply the
numbers. PET is then able to produce a data set containing power consumption
distributed over the simulation lifetime.

When estimating power usage for new architectures, it is hard to tell exactly
how to chose events and how expensive each event is. Therefore, we have to
assume that the new architecture is comparable to an old architecture, and thus
we can use the same power estimation numbers to give a new estimate.

\begin{figure}
    \includegraphics[width=0.9\textwidth]{figs/pet-workflow-gv.pdf}
    \caption{PET-pipeline}
    \label{fig:pipeline}
\end{figure}
