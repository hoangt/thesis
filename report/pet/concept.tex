\section{What is PET?}
\label{sec:whatispet}
\begin{figure}
    \includegraphics[width=0.9\textwidth]{figs/pet-workflow-gv.pdf}
    \caption{Generic usage of the PET program.}
    \label{fig:workflow}
\end{figure}

PET is a tool for power estimation of new as well as old architectures. It is
built by measuring existing hardware with great detail, capturing discrete events
and assigning each event a certain energy cost. When selected events have been
weighted, one can run the test program through a simulator set up to act as the
new hardware, as depicted \autoref{fig:workflow}. The simulator will generate a
tracelog containing the weighted events, and PET can then apply the numbers.
From this workflow, PET can produce a data set containing power consumption
distributed over the simulation lifetime -- a power profile of the program
execution. The new hardware will be weighted equally of a chosen existing
hardware, so this method requires a certain similarity beteen old and new
hardware. In general, all modern architectures contains equal principles
of function, and is thus mappable to each other.

The accuracy will indeed suffer the more it deviates from the reference
hardware, so it is assumed that the new architecture is comparable to an old
architecture. PET's primary use is to identify \emph{variations} between two
implementations of the same instruction set architecture. For instance, one can
experiment with a larger L2 cache to see how it affects energy usage (and
performance). The additional energy a larger L2 cache uses can be found from
existing implementations.

