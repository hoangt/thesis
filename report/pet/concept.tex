\section{What is PET?}
\label{sec:whatispet}
\begin{figure}
    \includegraphics[width=0.9\textwidth]{figs/pet-workflow-gv.pdf}
    \caption{Generic usage of the PET program.}
    \label{fig:workflow}
\end{figure}

PET is a tool for power estimation of new as well as old architectures. It is
built by measuring existing hardware with great detail, capturing discrete events
and assigning each event a certain energy cost. When selected events have been
weighted, one can run test programs through a simulator set up to act as the
new hardware, as depicted \autoref{fig:workflow}. The simulator will generate a
trace log containing the weighted events, which is then processed by PET.
From this workflow, PET can produce a data set containing power consumption
distributed over the simulation lifetime -- a power profile of the program
execution.

In case of estimating power for an unimplemented hardware platform, the new
hardware will be weighted similar to an existing implementation. As a
consequence, this method requires a certain similarity between old and new
hardware. We claim that in general, all modern computer architectures is built
from comparable principles, and thus mappable to each other.

The accuracy will indeed suffer the more it deviates from the reference
hardware. PET's primary use is to identify \emph{variations} between two
implementations of the same instruction set architecture. For instance, one can
experiment with a larger L2 cache to see how it affects energy usage (and
performance). The additional energy used by a larger L2 cache can be derived
from looking at existing implementations.

