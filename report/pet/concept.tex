\section{What is PET?}

\begin{figure}
    \includegraphics[width=0.9\textwidth]{figs/pet-workflow-gv.pdf}
    \caption{Workflow using PET}
    \label{fig:workflow}
\end{figure}

PET is built by measuring real hardware with great detail, capturing discrete
events and assigning each event a certain energy cost. As depicted in
\autoref{fig:workflow}, when selected events have been weighted, one can run the
test program through a simulator set up to act as the new hardware. The simulator
will generate a tracelog containing the weighted events, and PET can then apply
the numbers. From this workflow, PET can produce a data set containing power
consumption distributed over the simulation lifetime. As noted, the new hardware
will be weighted equally of a chosen existing hardware, so so this method
requires a certain similarity beteen the new and the old hardware. In general,
all traditional architectures contains equal principles of function, and is thus
mappable to each other, but accuracy will of course differ as hardware differs
in design, applied voltage, clock speed and process technology.

When estimating power usage for new architectures, it is hard to pick events and
assign a cost to it. We will have to assume that the new architecture is
comparable to an old architecture, so that we can use the old power estimation
numbers to guide a new estimate. For instance, one can experiment with a
larger L2 cache to see how it affects energy usage (and performance). The
additional energy a larger L2 cache uses can be found from existing
implementations.
