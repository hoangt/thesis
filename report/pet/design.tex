\section{Design}
When designing a tool for parsing large input files and applying flexible configuration alternatives,
there are quite a few aspects that needs to be considered. Through this section the design choises
briefly explained in \autoref{subsec:design_choises} will be further explained and more deeply understood.

\subsection{Input}
New hardware architectures are not easily analysed without physical
implementation, but often we are able to simulate its behaviour to acceptable
accuracy, and thus we are able to test different implementations with low cost.
These simulation runs can often be set up to produces trace logs which contains
a user controlled amount of detail. PET can use these trace logs and scan them
for predefined events, each affecting the power consumption of the simulated
hardware. Different simulators have different trace log formats and different
trace abilities. We have chosen GEM5 as our target simulator as it is easy to
configure, and trace is well implemented. As mentioned in
\autoref{subsec:design_choises} other options are available, but the support for
easily configurable CPU- and memory system along with the pre-implemented ARM
processors and in-department hands-on experience with this simulator made GEM5
the most logical choise.

When run with \texttt{--debug-flags=Bus,Cache,MemoryAccess,Exec} GEM5 will output trace files look like this:
\begin{lstlisting}
234320: system.cpu.dcache: Block addr 81f00 moving from state 0 to state: 7 (E) valid: 1
234320: system.cpu.dcache: Leaving recvTimingResp with ReadResp for address 81f00
234320: system.tol2bus.respLayer1: The bus is now busy from tick 234320 to 236376
116424: system.cpu T0 : 0x89d4.0  :   ldr   r1, [sp] #4        : MemRead :  D=0x00000000
116424: system.cpu T0 : 0x89d4.1  :   addi_uop   sp, sp, #4    : IntAlu :  D=0x00000000b
117012: system.cpu T0 : 0x89d8    :   mov   r2, sp             : IntAlu :  D=0x00000000b
117012: system.cpu T0 : 0x89dc.0  :   str   r2, [sp, #-4]!     : MemWrite :  D=0x0000000
117600: system.cpu T0 : 0x89dc.1  :   subi_uop   sp, sp, #4    : IntAlu :  D=0x00000000b
117600: system.cpu T0 : 0x89e0.0  :   str   r0, [sp, #-4]!     : MemWrite :  D=0x0000000
240000: system.membus: recvTimingResp: src system.membus.master[0] ReadResp 0x1640
240000: system.l2: Handling response to ReadResp for address 1640
240000: system.l2: Block for addr 1640 being updated in Cache
\end{lstlisting}

%weightfiles

\subsection{Output}

\subsection{Performance}

PET has to be designed for 

\begin{figure}
    \includegraphics[width=0.9\textwidth]{figs/pet-pipeline-gv.pdf}
    \caption{PET-pipeline}
    \label{fig:pipeline}
\end{figure}

put UML-figures here
trådpool
concurency
arbeidsdeling
ringbuffer, statisk vs. ikke statis størrelse


