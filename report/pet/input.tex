\section{Input}

PET needs two types of data in order to model power; a simulator trace log and
event-type weights. Optionally, PET can also read an annotation file and display
function calls in the output.

\subsection{gem5 Trace Logs}
In order to create the simulator trace log with required information, gem5 must
be run with a specific set of parameters. By executing gem5 with
\texttt{\textemdash\textemdash
debug-flags=Bus,\allowbreak{}Cache,\allowbreak{}MemoryAccess,\allowbreak{}Exec},
gem5 will output trace files that look like \autoref{lst:trace}.

\begin{lstlisting}[basicstyle=\tiny,caption={gem5 Trace
Log.},label={lst:trace},escapeinside={@}{@},float=htb]
@\label{line:physmem}@3021: system.physmem: Write of size 8 on address 0x82fe0 data 0xe1a0f00eee1d0f70
@\label{line:icache}@3021: system.cpu.icache: access for ReadReq address 9c0 size 64
@\label{line:cachemiss}@3021: system.cpu.icache: ReadReq (ifetch) 9c0 miss-
...
@\label{line:cacheupdate}@3432: system.cpu.dcache: Block addr 81f0 moving from state 0 to state:7 valid: 1
3432: system.cpu.dcache: Leaving recvTimingResp with ReadResp for address 81f00
3432: system.tol2bus.respLayer1: The bus is now busy from tick 234320 to 236376
@\label{line:memread}@1642: system.cpu T0 : 0x89d4.0 : ldr   r1, [sp] #4     : MemRead :  D=0x00000000
@\label{line:intalu}@1642: system.cpu T0 : 0x89d4.1 : addi_uop   sp, sp, #4 : IntAlu :  D=0x00000000b
1701: system.cpu T0 : 0x89d8   : mov   r2, sp          : IntAlu :  D=0x00000000b
1701: system.cpu T0 : 0x89dc.0 : str   r2, [sp, #-4]!  : MemWrite :  D=0x0000000
1760: system.cpu T0 : 0x89dc.1 : subi_uop   sp, sp, #4 : IntAlu :  D=0x00000000b
1760: system.cpu T0 : 0x89e0.0 : str   r0, [sp, #-4]!  : MemWrite :  D=0x0000000
4000: system.membus: recvTimingResp: src system.membus.master[0] ReadResp 0x1640
4000: system.l2: Handling response to ReadResp for address 1640
4000: system.l2: Block for addr 1640 being updated in Cache
\end{lstlisting}

Each line in \autoref{lst:trace} represents an event that happens in the
simulated hardware.  \autoref{line:physmem} tells that a write access to
physical memory has happened. \autoref{line:icache} is the event of instruction
cache access, while \autoref{line:cachemiss} shows that this request failed.
During this simulation, there is also events like \autoref{line:cacheupdate}
which represents that the data cache updates some content. The discrete
instructions running through the CPU is also logged, e.g. \autoref{line:memread}
shows a load instruction and \autoref{line:intalu} shows an add instruction.

Discrete events are extracted from the trace log and power consumption is
accumulated in equally sized timeslots in PET. Internally, these time steps are
called buckets and its size is parameter controlled (in terms of simulator
ticks). Often, it is more practical to specify the number of buckets in the
output rather then specifying the number of simulator ticks in each bucket.
PET is able to estimate the bucket size by peeking at the last tick of a trace
file. The trace file is not necessarily in tick order, but close enough to set
a reasonable bucket size. The bucket size estimation algorithm is shown in
\autoref{alg:bucket_size}.

\begin{algorithm}[htb]
    \caption{Bucket Size Detection Algorithm.}
    \label{alg:bucket_size}
    \begin{lstlisting}[language=Python,style=algo]
function numTicks( traceFile ):
    # Find file size
    eof_pos = traceFile.getSize()

    # Seek almost to end, avoid last newline
    traceFile.seek( eof_pos - 3 )

    # Trace from back of file to second last newline
    while not traceFile.currentChar is '\n':
        traceFile.seek_backwards

    # File stream position is now at beginning of last line
    # Parse this line
    simulatorEvent = parseLine( traceFile.getLine() )

    # Return the tick of the retreived event
    return simulatorEvent.getTick()
    \end{lstlisting}
\end{algorithm}

\autoref{lst:trace} shows that the events in the trace log is not necessarily in
their correct order. This means that PET must accumulate power consumption to the
entire timeline at all time. Consequently, we are unable to produce a continous
output flow, but have to store the results in memory and dump them when the
entire input is parsed.


\subsection{PET Weight Files}

Equally important as finding the correct events is assigning each event the
correct amount of power consumption. As each event will count differently
depending on the architecture, PET will read a weight file along with the gem5
trace log. A sample weight file is shown in \autoref{lst:weights_example}. As
the timeslots are specified in simulator ticks instead of CPU cycles, the values
have been chosen to match a 2~GHz processor, i.e. one CPU cycle per 500
simulator ticks \footnote{This is the default gem5 simulation granularity.}.
If you are applying this method to a processor with a different clock speed than
2~GHz, be aware that the weights has to be scaled proportionally. This is not
the case for the static power drain, as it is simply added to each timeslot, and
not scaled in accordance with bucket size. It is also important to understand
that the weight is applied once for each event, so events that naturally takes a
number of cycles will have a high weight which is in reality distributed over
many ticks. This will not be accurate if an expensive event is applied at the
border of a bucket, but we assume that accuracy at this level is not important
enough to increase the complexity of PET.

\lstinputlisting[caption={Weight File Example.},label={lst:weights_example},float,language=Python]{examples/weights.conf}

The weights displayed in \autoref{lst:weights_example} are assigned and
accumulated each time PET discovers a recognisable event in the log file. A
simplified version of this algorithm can be found in
\autoref{alg:power_accum_algo}

\begin{algorithm}
\caption{Power Accumulation Algorithm.}
\label{alg:power_accum_algo}
\begin{lstlisting}[language=Python,style=algo]
# map of accumulated power for each time step
map<time,power> output

# input is all trace log lines, elements in weight file and
# the determined bucket size (number of simulator ticks in
# each bucket)

function assignWeights( traceLogLines, weightMap, bucketSize )
    # run through each line
    for each line in traceLogLines:
        # extract event parameters from line
        simulatorEvent = parseLine( line )

        # get the assigned weight from weight file
        eventWeight = weightMap[simulatorEvent.getEventType()]

        # add this weight to the output map
        output[simulatorEvent.getTick()/bucketSize] += eventWeight
    return output
\end{lstlisting}
\end{algorithm}

\subsection{Program Annotation Files}
\label{subsec:annot}

PET has the ability to annotate its output using a map from PC to function name
(or rather, symbol name). The simulated binary itself is not an input to PET,
instead PET comes bundled with an annotation tool: \texttt{scripts/annotate.sh}.
This tool extracts symbols from the binary file, compiled with debugging
symbols, to a text file in the format seen in \autoref{lst:annot}. The left
column represents the address where the function is found, and the right
column is the function name. PET will tag each bucket with the last seen
symbol within that bucket. \texttt{scripts/annotate.sh} is constructed by using
\texttt{objdump} and is printed in \autoref{lst:annotscript}.

\lstinputlisting[caption={Annotation File Example.},label={lst:annot},float,language=Python]{examples/annot.conf}
\lstinputlisting[float=htb,label={lst:annotscript},caption={\texttt{scripts/annotate.sh}:
Script to extract symbols from a binary.},language=sh]{../pet/scripts/annotate.sh}

It should be mentioned that a program compiled with debugging symbols contains
hundreds, if not thousands of symbols. Often, many of these symbols are called
in clusters, while long periods of the program are spent in loops that are not
using any symbols at all. This often renders a graphical representation of the
power log with complete annotation as a complete mess. We have found it to be a
good recommendation to create the annotation file using the mentioned tools, but
manually filtering out only the symbols of interest.

