\pagestyle{empty}
\renewcommand{\abstractname}{Sammendrag}
\begin{abstract}
\noindent Sikkerheten til nesten all offentlig nøkkel-kryptografi er basert på et vanskelig beregnbarhetsproblem. Mest velkjent er problemene med å faktorisere heltall i sine primtallsfaktorer, og å beregne diskrete logaritmer i endelige sykliske grupper. I de to siste tiårene, har det imidlertid dukket opp en rekke andre offentlig nøkkel-systemer, som baserer sin sikkerhet på helt andre type problemer. Et lovende forslag, er å basere sikkerheten på vanskeligheten av å løse store likningsett av flervariable polynomlikninger. En stor utfordring ved å designe slike offentlig nøkkel-systemer, er å integrere en effektiv ``falluke'' (trapdoor) inn i likningssettet. En ny tilnærming til dette problemet ble nylig foreslått av Gligoroski m.f., hvor de benytter konseptet om kvasigruppe-strengtransformasjoner (quasigroup string transformations). I denne masteroppgaven beskriver vi en metodikk for å identifisere sterke og svake nøkler i det nylig foreslåtte multivariable offentlig nøkkel-signatursystemet MQQ-SIG, som er basert på denne idéen.

Vi har gjennomført et stort antall eksperimenter, basert på Gröbner basis angrep, for å klassifisere de ulike parametrene som bestemmer nøklene i MQQ-SIG. Våre funn viser at det er store forskjeller i viktigheten av disse parametrene. Metodikken består i en klassifisering av de forskjellige parametrene i systemet, i tillegg til en innføring av konkrete kriterier for hvilke nøkler som bør velges. Videre, har vi identifisert et unødvendig krav i den originale spesifikasjonen, som krevde at kvasigruppene måtte oppfylle et bestemt kriterie. Ved å fjerne denne betingelsen, kan nøkkel-genererings-algoritmen potensielt øke ytelsen med en stor faktor. Basert på alt dette, foreslår vi en ny og forbedret nøkkel-genereringsalgoritme for MQQ-SIG, som vil generere sterkere nøkler og være mer effektiv enn den originale nøkkel-genereringsalgoritmen.  
\end{abstract}
