\section{Results}

There are several factors that together makes up the total performance of PET. First, the simulator must be
configured as close to the real hardware as time and knowledge permits. Next, one must find power weights that
seems close to those of the realized hardware. In this particular research, the processor is already existing,
and the simulator, PET and the strategy of the gentetic algorithm will determine how close power consumption
can be estimated. This section will take a closer look to each of these parameters.

\subsection{gem5 CPU model accuracy}

After modelling different configurations for gem5, accuracy as depicted in \autoref{tbl:gem5runtimeaccuracy}
was achieved. Each test was run with the command written out in \autoref{lst:gem5commandline}, with \texttt{CPU}
changed with  \texttt{exynos\_4412p}, \texttt{arm\_detailed} and \texttt{timing}.

\begin{lstlisting}[language=sh,label={lst:gem5commandline},caption={gem5 Command Line}]
$ build/ARM/gem5.opt --remote-gdb-port=0 -d m5out-time/sha2-sha2
    configs/example/se.py -c bin/sha2/sha2 --cpu-type=CPU
    --mem-type=LPDDR2_S4_800_x32 --sys-clock=440MHz
    --cpu-clock=1700MHz --num-l3caches=0 --caches --l2cache
    --l2_assoc=16 --l2_size=1MB --l1d_size=32kB
    --mem-size=2048MB --l1d_assoc=4 --l1i_assoc=4
\end{lstlisting}


% Move table to result chapter
\begin{table}
\centering
\begin{tabular}{|l|c|c|c|c|}
\hline
 & add-add & pi-pi & sha2-sha2 & trend-trend \\
\hline
Real hardware & 0.017600  & 0.013500 & 0.022600 & 0.014600 \\
gem5 modified O3    & 0.017541 & 0.013790 & 0.022819 & 0.011898 \\
gem5 original O3    & 0.008777 & 0.006708 & 0.010368 & 0.004344 \\
gem5 timing simple  & 0.035039 & 0.019564 & 0.040659 & 0.020503 \\
\hline
\end{tabular}
\caption{gem5 runtime accuracy (O3 with classic memory system)}
\label{tbl:gem5runtimeaccuracy}
\end{table}




\subsection{GA optimization}
As PET has been optimized by a genetic inspired algorithm, it is likly to perform very well on the data sets it
has been optimized for. When such training is done, a controll test is needed in order to verify that the result
is good for the general problem solved, not only the specific instances used for training.

\subsection{Training}

The training sets are displayed in \autoref{fig:trend-training},
\autoref{fig:pi-training}, \autoref{fig:sha2-training} and
\autoref{fig:add-training}

\begin{figure}[ht]
\centering
\includegraphics[width=\textwidth]{figs/training/trend-trend.pdf}
\caption{Overlay of PET training results (red) and training data (green)}
\label{fig:trend-training}
\end{figure}
\begin{figure}[ht]
\centering
\includegraphics[width=\textwidth]{figs/training/pi-pi.pdf}
\caption{Overlay of PET training results (red) and training data (green)}
\label{fig:pi-training}
\end{figure}
\begin{figure}[ht]
\centering
\includegraphics[width=\textwidth]{figs/training/sha2-sha2.pdf}
\caption{Overlay of PET training results (red) and training data (green)}
\label{fig:sha2-training}
\end{figure}
\begin{figure}[ht]
\centering
\includegraphics[width=\textwidth]{figs/training/add-add.pdf}
\caption{Overlay of PET training results (red) and training data (green)}
\label{fig:add-training}
\end{figure}

