\pagestyle{empty}
\begin{abstract}

    \noindent Energy efficiency is currently one of the most challenging goals
    of modern computer design. High power density limits further performance
    growth, and energy efficiency affects both the power bill for supercomputers
    and battery lifetime for embedded devices. In this thesis we investigate
    energy consumption and architectural properties of an ARM Cortex-A9 and
    create a tool for estimating its power consumption through simulation.

    Instruction level energy consumption is determined through measurements and
    experiments done on real hardware, which are further mapped to architectural
    events found in the gem5 simulator. Our tool utilizes these events together
    with a simulator trace log and outputs a representation of energy
    consumption over time.

    Our method can be applied during the development process at the simulator
    level, while traditional methods typically involves hardware synthesis. The
    results shows that our tool can estimate energy consumption within 10~\% on
    general workloads, and is able to identify power consumption trends
    throughout a program.

\end{abstract}
