\pagestyle{empty}
\begin{abstract}

\noindent Since the beginning of semiconductor technology, transistor size has
decreased and are still decreasing with a tremendous rate. Computer engineers
and architects have utilized higher transistor count to create more advance and
more dense computer chips, but are now facing problems with power consumption
and heat dissipation. In this thesis, an ARM-based SoC is studied in great
detail and is simulated using the gem5 architectural simulator. Using a genetic
algorithm, different power consuming events was mapped to a fixed energy
contribution.

Similar to measuring the execution of a program on a real processor using an
ammeter, we have built a tool called PET which provides the same for a
simulated program on unrealized hardware. PET parses trace logs from the
simulator and applies the energy contribution from each event. The output is
given as a textual or graphical timeline with current drain plotted at each data
point.

Comparing to other tools which does much the same, PET works much earlier in the
design stage, and can be applied without knowledge about process technology, RTL
or physical layout of a chip. This tool can help engineers figure out how a new
feature or a change in an existing chip will influence energy efficiency.
Benchmarks and test data indicates that PET is able to predict power consumption
within 10\% of the real value, and given the abstraction level which PET works,
any more accurate would needed more information. PET will give an indication of
better or worse, not an exact prediction of energy consumption.

\end{abstract}
