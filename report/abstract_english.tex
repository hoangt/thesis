\pagestyle{empty}
\begin{abstract}

    \noindent Energy efficiency is currently one of the biggest challenges in
    modern computer design. High power density limits further performance
    growth, and energy efficiency affects both the power bill for supercomputers
    and battery lifetime for embedded devices. A better understanding of energy
    efficiency during the design stage eases development of new architectures.
    In this thesis we investigate energy consumption and architectural
    properties of an ARM Cortex-A9 processor. Further, this information is used
    to create a tool for estimating its power consumption through simulation.

    Instruction level energy consumption is determined through measurements and
    experiments on real hardware, which are further mapped to certain
    architectural events found in the gem5 simulator. The tool utilizes these
    events together with a simulator trace log and outputs a representation of
    energy consumption over time.

    This method can be applied during the development process at the simulator
    level, while traditional methods typically involves hardware synthesis.

    The results show that this tool can estimate energy consumption with margin
    of error of 5~\% on general workloads, and is able to identify power
    consumption trends throughout a program.

\end{abstract}
