\section{About the Test Environment}

PET has already proven that it is able to do the job of power estimation for the
training data. The set of workloads has been split in a training set as
presented in \autoref{sec:workloads} and a test set. This separation is
necessary in order to achieve a good result from the genetic algorithm
\cite{russellnorvig,rajer2003separation}.

The test data set consists of a short Dhrystone-run and a fictional add-loop,
each representing very different use of the processor. The Dhrystone test aims
to measure overall performance of the general processor, while the add-loop will
stress the ALUs leaving most of the other parts of the CPU at idle. The
Dhrystone test is run for 100000 iterations, which is too short for a real
Dhrystone performance benchmark, but it is becomes long enough to measure
current drain and short enough to simulate on a simple workstation.

There are sources claiming that gem5 is very accurate 
\cite{butko2012accuracy,pusdesrissources}, but most of these claims are done
with focus on overall performance of long-running benchmarks, not the correctness
of the architectural events. This renders as a problem for PET, which needs
correct architectural events in order to calculate the power profile. Yet
another important recap is that the gem5 processor model is not identical, but
merely similar to the Exynos-4412. Properties like the fast-loop mode is not implemented
in gem5 and parameters for the branch-predictor is not publicly available,
meaning that they are most likely not correct in the simulated model. Further,
this means that some discrepancy between the measurements and PETs prediction is
natural.
