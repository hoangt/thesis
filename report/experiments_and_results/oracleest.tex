\section{Description of the ultimate power estimator}
The goal of any power estimator tool would be deliver the correct answer
of how much power a certain architecture would use, but there are many
caveats making this almost impossible with our current methods.

A simple and non-complete list of things that are important with respect to power consumption,
but are very hard to get correct in a simulator are described in the following list:
    \begin{description}
    \item[Cache and memory state]
        The cache and memory systems, together with all sorts of cache prefetchers, branch predictors,
        fast-loop queues and so on will make it difficult to assure that the simulated system are in
        the exact same state as the physical system.
    \item[Interrupts]
        It is very hard to predict when interrupts will come. Interrupts mostly cause context switches,
        and thus a change in system behaviour and power drain.
    \item[CPU and system specifications not disclosed to tester]
        Unless the whole system is built in-house, there will often be certain specifications that
        are not available to the tester. PET was built against a CPU where some details where not disclosed,
        and thus it was not trivial to find the most likly details to feed to the simulator.
    \item[Real world implementation vs. simulated architecture]
        A simulated system will in most cases resemble ideas, not the exact implementation. This means
        that one can test if an idea work, but it is hardly assumed that the simulated system is exactly
        equal to any real world system.
    \end{description}
This presented list is by no means a show stopper, but tells us
that we need to focus on something that is good enough. The best power estimator that could
ever be built around a simulator would be the one that reflected the ideas of the simulated
hardware in the best way possible. Such an estimator would also give a hint about final power
consumption and its trend over a set of test programs, and would such be usable for application
specific power optimization as well as general power optimizations. PET aims to be applicable for
both these concepts.
